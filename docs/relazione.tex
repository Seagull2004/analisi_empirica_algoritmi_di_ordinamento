\documentclass[a4paper, 12pt, oneside]{book}
\usepackage[italian]{babel} % Lingua italiana
\usepackage{authblk} % Per gli autori

\renewcommand\Authands{ e }

\title{Analisi empirica degli algoritmi di ordinamento}

\author[ ]{Graziano Francesco \thanks{Email: graziano.francesco@spes.uniud.it, Matricola: 166680}}
\author[ ]{Ongaro Michele \thanks{Email: ongaro.michele@spes.uniud.it, Matricola: 168049}}
\author[ ]{Petri Riccardo \thanks{Email: petri.riccardo@spes.uniud.it, Matricola: 167623}}
\author[ ]{Ungaro Marco \thanks{Email: ungaro.marco@spes.uniud.it, Matricola: 168934}}

\affil[ ]{Università degli Studi di Udine, Dipartimento di Matematica e Informatica}

\date{A.A. 2024-2025}
%\date{}

\usepackage{amsmath} % Per equazioni avanzate
\usepackage{amssymb} % Simboli matematici
\usepackage{graphicx} % Per immagini
\usepackage{lmodern} % Font simile a Computer Modern
\usepackage{geometry} % Per margini personalizzati
\usepackage{listings} % scrivere codice

\geometry{a4paper, margin=2.5cm}
\cleardoublepage

\begin{document}

\renewcommand{\contentsname}{Contenuti}
\renewcommand{\chaptername}{Capitolo}

\maketitle % copertina del documento
\tableofcontents % sommario di tutti i titoli

\chapter{Introduzione}\label{chap:Introduzione} % (fold)

Il progetto richiede l'implementazione di quattro algoritmi di ordinamento per array interi di dimensioni variabili.
Gli algoritmi che andremo ad analizzare sono il Counting Sort, il Quick Sort, il Quick Sort 3 way e il Radix Sort (algoritmo a scelta).
Oltre alla corretta implementazione viene richiesto di effettuare un analisi empirica dei tempi medi di esecuzione degli algorimti al variare della dimensione dell'array e del range dei valori interi.
Per stimare i tempi di esecuzione di questi algoritmi garanantendo un errore relativo massimo pari a 0.001 adotteremo le seguenti metodologie:
\begin{itemize}
    \item Utilizzeremo un clock di sistema monotono per garantire precisione nelle misurazioni (ad esempio, \texttt{perf\_counter()} del modulo \(time\) in Python);
    \item Andremo a generare almeno 100 campioni per ciascun grafico, con i valori dei parametri (dimensione dell’array \(n\) e intervallo dei valori \(m\)) distribuiti secondo una progressione geometrica;
    \item Effettueremo più esecuzioni per ogni campione, per stimare in modo affidabile il tempo medio di esecuzione e, eventualmente, il relativo errore.
\end{itemize}
Dopo aver stimato i tempi di esecuzione per ciascun algoritmo, risulterà interessante confrontare i grafici ottenuti per analizzare il comportamento degli algoritmi in diverse situazioni, come il caso peggiore, quello migliore o in quello medio.

Da tutto questo potremmo ottenere una verifica empirica dell'andamento asintotico dei tempi di esecuzione di ogni algoritmo.

% chapter Introduzione (end)

\chapter{Counting Sort}\label{chap:Counting Sort} % (fold)

Spiegazione dell'algoritmo di ordinamento

Analisi della complessità

Grafico dei tempi di esecuzione

% chapter Counting Sort (end)

\chapter{Quick Sort}\label{chap:Quick Sort} % (fold)

% chapter Quick Sort (end)

\chapter{Quick Sort 3 way}\label{chap:Quick Sort 3 way} % (fold)

% chapter Quick Sort 3 way (end)

\chapter{Radix Sort}\label{chap:Radix Sort} % (fold)

% chapter Radix Sort (end)

\chapter{Conclusioni}\label{chap:Conclusioni} % (fold)

% chapter Conclusioni (end)


\end{document}

